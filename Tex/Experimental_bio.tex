\subsection{Biological testing\label{sec:exp_bio}}

Compounds were tested against \textit{P. aeruginosa} PAO1\cite{Stover2000} and YM64\cite{Morita2001}.
C$_4$-HSL \compound{cmpd:HL4}, HHQ \compound{cmpd:HHQ}, PQS \compound{cmpd:PQS}, ciprofloxacin \compound{cmpd:Cip}, trimethoprim \compound{cmpd:Tri} and DMSO were included as controls, along with LB to check for contamination of the plates.

The first set of autoinducer-antibiotic conjugates (see \ref{sec:bio1}) were tested at 2, 1, 0.5, 0.25, 0.125 and 0.0625 $\mu$M. 
Breathe-Easy$^{\tiny{\textregistered}}$ sealing membranes from Diversified Biotech were used and the plates were placed without lids in a open box containing tissue paper wetted with distilled water in order to control evaporation. 
OD readings at 595 nm were taken at 5 and 24 h, and biofilm quantification was carried out soon after the 24 h OD reading. Crystal violet-stained plates were also read at 595 nm.
Only a 5 h OD reading in YM64 was obtained for the cleavable HSL-Cip conjugates.

The HSL analogue-Cip(Me) conjugates (see \ref{sec:bio2}) were tested at 25, 2, 1, 0.5, 0.25 and 0.125 $\mu$M in triplicate.
AeraSeal$^{\tiny{\texttrademark}}$ films from Excel Scientific were used. A plate lid was used, but the humidified box was not.
OD readings at 600 nm were taken at 0, 1, 2, 3, 4, 5, 6, 7, 8, 24 and 48 h. 
Biofilm inhibition testing was carried out on plates grown for 24 and 48 h. Biofilm dispersal testing was carried out by growing plates for 24 h, followed by addition of the compounds, incubation for a further 24 h and quantification of the biofilms. Crystal violet-stained plates were read at 550 nm.

\subsubsection{Antibiotic susceptibility\label{sec:ABsus}}

Antibiotic susceptibility was determined using spectrophotometry measurements.
Colonies of the desired strains were grown at 37 $^{\circ}$C overnight on LB agar.
The colonies were used to inoculate LB (10 mL) and these cultures were grown at 37 $^{\circ}$C overnight. 
The cultures were diluted 1/100 with LB, and 99 $\mu$l diluted culture per well was added to Nunclon$^{\tiny{\textregistered}}$ flat-bottomed clear 96-well plates. 1 $\mu$l of compound solution in DMSO was then added from master plates and the plates were covered with adhesive aeration filters. 
The plates were shaken at 37 $^{\circ}$C and 100 rpm and OD was recorded periodically using a Biochrom EZ Read 400 microplate reader.

\subsubsection{Quantification of biofilms\label{sec:biofilm_quant}}

Biofilms were quantified using a method described previously\cite{OToole1998,Li2015a}.
After the bacteria had grown for the desired amount of time, the culture was aspirated out of the wells using a pipette tip attached to a vacuum pump, making sure not to touch the sides of the wells. Water (120 $\mu$l) was then added and aspirated out again. This process was repeated twice more to thoroughly wash out planktonic cells. Crystal violet (120 $\mu$l, 0.1\% $m$/$v$) was added and left for 15 min, then aspirated out. The wells were washed again with water (3 $\times$ 120 $\mu$l). Acetic acid (120 $\mu$l, 30\% $v$/$v$ aq.) was added and left for 15 min then the plate was vortexed and read using a Biochrom EZ Read 400 microplate reader at 595 nm.

\subsubsection{Biofilm inhibition}

The plates were prepared as in \ref{sec:ABsus}. The plates were shaken at 37 $^{\circ}$C and 100 rpm for 24 h followed by quantification of biofilm growth as shown in \ref{sec:biofilm_quant}.

\subsubsection{Biofilm dispersal \label{sec:disp}}

The plates were prepared as in \ref{sec:ABsus}, initially without the addition of compound solutions. The box of plates was shaken at 37 $^{\circ}$C and 100 rpm for 24.
1 $\mu$l of compound solution in DMSO was then added to each well from master plates and the plates were shaken as above for a further 24 h followed by measurement of OD and quantification of biofilm growth as shown in \ref{sec:biofilm_quant}.

