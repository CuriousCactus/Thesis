\subsection{Future work \label{sec:Fut2}}

\subsubsection{HSL analogue derivatives}

\todo{HL4CipMe}
\todo{HSL analogue cip conjugates with no Me}

A selection head groups which could be used in future conjugates are shown in \ref{fgr:fut_heads}. These have all been shown to modulate HSL-mediated quorum sensing as part of acyl-HSLs\cite{Smith2003a,Welch2005,Ishida2007,Olsen2002,Smith2003,Hodgkinson2012a,Marsden2010}. The most obvious targets are the cyclopentanone derivatives, as this could be synthesised from the alcohols above. The aniline, pyridine, quinoline and cyclopentyl amine head groups are commercially available and hence derivatives of these could be easily obtained. The 3- and 4-substituted HSL analogues require synthesis, but a convenient route has been devised\cite{Olsen2002}.

\begin{figure}[H]
	\begin{center}
		\includegraphics[scale=1]{fut_heads}
		\caption{HSL analogue head groups for use in future conjugates.
		\label{fgr:fut_heads}}
	\end{center}
\end{figure}

\subsubsection{Biology}

\todo{Repeat biofilm stuff}

Ganguly \textit{et al.} used Bac-Light Live/Dead staining and confocal microscopy to image the biofilms, whereas so far I have used crystal violet staining. Crystal violet does not differentiate between live or dead cells, and so might not pick up on the antibacterial effects of compounds. However, their confocal microscopy results show a quantifiable decrease in biofilm thickness, and it may be possible to detect this using crystal violet.

\todo{XTT stain}