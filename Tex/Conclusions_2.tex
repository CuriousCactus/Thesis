\subsection{Conclusions}

\subsubsection{Library synthesis}

In this section, a library of HSL analogue-ciprofloxacin conjugates was successfully synthesised and tested for antibiotic activity.
A range of 7 head groups (see \ref{sec:heads}) and two linking strategies were used.
Unfortunately the branching route that was initially proposed (see \ref{sec:branch}) was not feasible for the alcohol-containing head groups and was low yielding for others, probably due to internal cyclisation (this side reaction could with hindsight be avoided by changing the linker length).

Given the difficulties in the branching synthesis, routes to the differently-linked compounds were optimised separately: the alkyl-linked conjugates were best formed using peptide coupling and the triazole-linked conjugates via a chloride intermediate. 
Direct comparisons of routes are not possible without repeating syntheses, but if it is assumed that peptide coupling of homocysteine thiolactone hydrochloride \compound{cmpd:SHLHCl} to carboxylic acid \compound{cmpd:HOO4CipMeTFA} would have a similar yield to the coupling with (1\textit{R},2\textit{R})-2-aminocyclopentan-1-ol \compound{cmpd:HOcy5NH2_RR}, approximate comparisons can be made.
The synthesis of the HCTL-CipMe conjugate \compound{cmpd:SHL4CipMe} described in \ref{sec:HCTL} has an overall yield of 11\%, whereas the route to the cyclopentanol-CipMe conjugate \compound{cmpd:HOcy5NH4CipMe_SS} shown in \ref{sch:HOcy5NH4CipMe_RR_synth} has an overall yield of 26\%. Moreover, if the yield starting from the head group is considered, the yield is 55\% vs. 11\%.
Therefore, the peptide coupling route is recommended for further investigation if the alkyl-linked library is to be expanded.
%chloride

Synthesis of the azido autoinducer analogues via the chloride is also recommended as the bromide is thought to cyclise readily (this could explain the poor yields of the 2- and 3-methoxybenzene derivatives).

Preparative HPLC was identified as the best purification method for these conjugates (note that the standard acidic method used hydrolyses the lactone of native HSL and so cannot be used in that case).


\subsubsection{Biology}

Biofilms can drastically increase MIC for many antibiotics \cite{Ceri1999}. For ciprofloxacin in \textit{P. aeruginosa} the MIC increases by 16 fold. 

Ganguly \textit{et al.} \cite{Ganguly2011} found the MICs of ciprofloxacin and a BHL analogue-ciprofloxacin \compound{cmpd:SHL4CipMe} conjugate under standard planktonic conditions by introducing the compounds to liquid culture. The MICs were found to be ten times lower for ciprofloxacin vs. the conjugate \compound{cmpd:SHL4CipMe} (5 vs 50 um). They then investigated the effect of the compounds on biofilms. The compounds were first cultured at 25um, with PA liquid culture. As expected, the culture failed to grow and form biofilm in the presence of ciprofloxacin, but did grow in the presence of the conjugate \compound{cmpd:SHL4CipMe}. They then cultured biofilm for 24 hours before adding the compounds, and found that, in contrast, the conjugate \compound{cmpd:SHL4CipMe} disrupted the biofilm more effectively than ciprofloxacin. When the biofilm was cultured for 48 or 72 hours the conjugate similarly disruptive effects, whereas ciprofloxacin 'did not show any significant antibacterial activity'.

Ganguly \textit{et al.} used Bac-Light Live/Dead staining and confocal microscopy to image the biofilms, whereas so far I have used crystal violet staining. Crystal violet does not differentiate between live or dead cells, and so might not pick up on the antibacterial effects of compounds. However, their confocal microscopy results show a quantifiable decrease in biofilm thickness, and it may be possible to detect this using crystal violet.

The conjugate \compound{cmpd:SHL4CipMe} developed by Ganguly \textit{et al.} contained a thiolactone AHL. The unconjugated thiolactone BHL \compound{cmpd:SHL4} was shown to have 'either enhanced uptake or functional activity' when compared with BHL \compound{cmpd:HL4}. Therefore it seems possible that my compounds may not show enhanced antibiotic activity, where thiolactone analogues might.



Ganguly \textit{et al.} \cite{Ganguly2011} found the MICs of ciprofloxacin and a BHL analogue-ciprofloxacin conjugate \compound{cmpd:SHL4CipMe} under standard planktonic conditions by introducing the compounds to liquid culture. The MICs were found to be ten times lower for ciprofloxacin vs. the conjugate \compound{cmpd:SHL4CipMe} (5 vs 50 $\mu$M). They then investigated the effect of the compounds on biofilms. The compounds were first cultured at 25 $\mu$M, with PA liquid culture. As expected, the culture failed to grow and form biofilm in the presence of ciprofloxacin, but did grow in the presence of the conjugate \compound{cmpd:SHL4CipMe}. They then cultured biofilm for 24 hours before adding the compounds, and found that, in contrast, the conjugate \compound{cmpd:SHL4CipMe} disrupted the biofilm more effectively than ciprofloxacin. When the biofilm was cultured for 48 or 72 h the conjugate similarly disruptive effects, whereas ciprofloxacin 'did not show any significant antibacterial activity'.
