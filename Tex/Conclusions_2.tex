\subsection{Conclusions}

\subsubsection{Library synthesis}

In this section, a library of HSL analogue-ciprofloxacin conjugates was successfully synthesised and tested for antibiotic activity.
A range of 7 head groups (see \ref{sec:heads}) and two linking strategies were used.
Unfortunately the branching route that was initially proposed (see \ref{sec:branch}) was not feasible for the alcohol-containing head groups and was low yielding for others, probably due to internal cyclisation (this side reaction could with hindsight be avoided by changing the linker length).

Given the difficulties in the branching synthesis, routes to the differently-linked compounds were optimised separately: the alkyl-linked conjugates were best formed using peptide coupling and the triazole-linked conjugates via a chloride intermediate. 
Direct comparisons of routes are not possible without repeating syntheses, but if it is assumed that peptide coupling of homocysteine thiolactone hydrochloride \compound{cmpd:SHLHCl} to carboxylic acid \compound{cmpd:HOO4CipMeTFA} would have a similar yield to the coupling with (1\textit{R},2\textit{R})-2-aminocyclopentan-1-ol \compound{cmpd:HOcy5NH2_RR}, approximate comparisons can be made.
The synthesis of the HCTL-CipMe conjugate \compound{cmpd:SHL4CipMe} described in \ref{sec:HCTL} has an overall yield of 11\%, whereas the route to the cyclopentanol-CipMe conjugate \compound{cmpd:HOcy5NH4CipMe_SS} shown in \ref{sch:HOcy5NH4CipMe_RR_synth} has an overall yield of 26\%. Moreover, if the yield starting from the head group is considered, the yield is 55\% vs. 11\%.
Therefore, the peptide coupling route is recommended for further investigation if the alkyl-linked library is to be expanded.
%chloride

Synthesis of the azido autoinducer analogues via the chloride is also recommended as the bromide is thought to cyclise readily (this could explain the poor yields of the 2- and 3-methoxybenzene derivatives).

Preparative HPLC was identified as the best purification method for these conjugates (note that the standard acidic method used hydrolyses the lactone of native HSL and so cannot be used in that case).


\subsubsection{Biology}

\subsubsubsection{Controls}

\subsubsubsection{Cip vs CipMe}

The Cip triazole conjugates had higher activity than the CipMe conjugates. This was mirrored in the controls: CipMe\compound{cmpd:CipMe} showed very little activity compared to ciprofloxacin \compound{cmpd:Cip}. It was assumed that methyl ciprofloacin would act as a prodrug, and was used in the Ganguly conjugate, but this was apparently not the case. The methylated controls also didn't show any activity, but Y4Cip did.

\subsubsubsection{Best ones}

The most promising compounds were \compound{cmpd:SHL4THCip}, \compound{cmpd:2MeOA4T4Cip} and \compound{cmpd:3MeOA4T4Cip}.

PA doesn't develop resistance to \compound{cmpd:SHL4THCip} over the course of a 48 h incubation whereas Cip does \cite{Su2010}.

\subsubsubsection{biofilm fail}

Biofilm inhibition and dispersal were measured using crystal violet staining  (see \ref{sec:exp_bio}). Unfortunately the results were largely unreliable as it was obvious that biofilm was growing more in the wells at the edges of the plates, and that this was overwhelming any other trends. This effect was probably due to increased evaporation from the outer wells. 
It is likely that this effect was seen in these results, but not those in \ref{sec:bio1}, due to a change in the conditions that the plates were incubated in. Specifically, a different type of plate seal was used and a humid environment was maintained in the incubator (see \ref{sec:exp_bio}) in the previous experiments.

As biofilm formation is induced by hypoxia\cite{Ghotaslou2013} (which might occur in the centre of the plate when a plastic lid was used in addition to the adhesive plate seal) this cannot account for the increased biofilm growth at the edges of the plate. Neither can the increased concentration of NaCl that would occur due to evaporation of water from the edges of the plate, as this too decreases biofilm formation\cite{Bazire2007}. \todo{culture conc?}
A reasonable explanation could be that evaporation would leave a residue of dried planktonic cells the edges of the wells which would be stained by the crystal violet.
To avoid this problem in the future the plates should be incubated in a humid atmosphere without a plastic plate lid.
