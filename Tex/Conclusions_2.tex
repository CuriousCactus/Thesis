\subsection{Conclusions}

In this section, a library of HSL analogue-ciprofloxacin conjugates was successfully synthesised and tested for antibiotic activity.
A range of 7 head groups (see \ref{sec:heads}) and two linking strategies were used.
Unfortunately the branching route that was initially proposed (see \ref{sec:branch}) was not feasible for the alcohol-containing head groups and was low yielding for others, probably due to internal cyclisation (this side reaction could with hindsight be avoided by changing the linker length).

Given the difficulties in the branching synthesis, routes to the differently-linked compounds were optimised separately: the alkyl-linked conjugates were best formed using peptide coupling and the triazole-linked conjugates via a chloride intermediate. 
Direct comparisons of routes are not possible without repeating syntheses, but if it is assumed that peptide coupling of homocysteine thiolactone hydrochloride \compound{cmpd:SHLHCl} to carboxylic acid \compound{cmpd:HOO4CipMeTFA} would have a similar yield to the coupling with (1\textit{R},2\textit{R})-2-aminocyclopentan-1-ol \compound{cmpd:HOcy5NH2_RR}, approximate comparisons can be made.
The synthesis of the HCTL-CipMe conjugate \compound{cmpd:SHL4CipMe} described in \ref{sec:HCTL} has an overall yield of 11\%, whereas the route to the cyclopentanol-CipMe conjugate \compound{cmpd:HOcy5NH4CipMe_SS} shown in \ref{sch:HOcy5NH4CipMe_RR_synth} has an overall yield of 26\%. Moreover, if the yield starting from the head group is considered, the yield is 55\% vs. 11\%.
Therefore, the peptide coupling route is recommended for further investigation if the alkyl-linked library is to be expanded.
%chloride

Synthesis of the azido autoinducer analogues via the chloride is also recommended as the bromide is thought to cyclise readily (this could explain the poor yields of the 2- and 3-methoxybenzene derivatives).

Preparative HPLC was identified as the best purification method for these conjugates (note that the standard acidic method used hydrolyses the lactone of native HSL and so cannot be used in that case).



