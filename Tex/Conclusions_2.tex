\subsection{Conclusions}

\subsubsection{Library synthesis}

In this section, a library of HSL analogue-Cip(Me) conjugates was successfully synthesised and tested for antibiotic activity.
A range of 7 head groups (see \ref{sec:heads}) and two linking strategies were used.
Unfortunately the branching route that was initially proposed (see \ref{sec:branch}) was not feasible for the alcohol-containing head groups and was low yielding for others, probably due to internal cyclisation (this side reaction could with hindsight be avoided by changing the linker length).

Given the difficulties in the branching synthesis, routes to the differently-linked compounds were optimised separately: the alkyl-linked conjugates were best formed using peptide coupling and the triazole-linked conjugates via a chloride intermediate. 
Direct comparisons of routes are not possible without repeating syntheses, but if it is assumed that peptide coupling of homocysteine thiolactone hydrochloride \compound{cmpd:SHLHCl} to carboxylic acid \compound{cmpd:HOO4CipMeTFA} would have a similar yield to the coupling with (1\textit{R},2\textit{R})-2-aminocyclopentan-1-ol \compound{cmpd:HOcy5NH2_RR}, approximate comparisons can be made.
The synthesis of the HCTL-CipMe conjugate \compound{cmpd:SHL4CipMe} described in \ref{sec:HCTL} has an overall yield of 11\%, whereas the route to the cyclopentanol-CipMe conjugate \compound{cmpd:HOcy5NH4CipMe_SS} shown in \ref{sch:HOcy5NH4CipMe_RR_synth} has an overall yield of 26\%. Moreover, if the yield starting from the head group is considered, the yield is 55\% vs. 11\%.
Therefore, the peptide coupling route is recommended for further investigation if the alkyl-linked library is to be expanded.
%chloride

Synthesis of the azido autoinducer analogues via the chloride is also recommended as the bromide is thought to cyclise readily (this could explain the poor yields of the 2- and 3-methoxybenzene derivatives).

Preparative HPLC was identified as the best purification method for these conjugates (note that the standard acidic method used hydrolyses the lactone of native HSL and so cannot be used in that case).


\subsubsection{Biology}

The ciprofloxacin triazole conjugates had higher activity than the methyl ciprofloxacin conjugates. This was mirrored in the controls: methyl ciprofloxacin \compound{cmpd:CipMe} showed little activity compared to ciprofloxacin \compound{cmpd:Cip}. It was assumed that methyl ciprofloacin \compound{cmpd:CipMe} would act as a prodrug, as the HCTL-CipMe conjugate \compound{cmpd:SHL4CipMe} synthesised by Ganguly \textit{et al.}\cite{Ganguly2011} was a methyl ester and showed activity, but these results suggest otherwise. However, the HCTL-CipMe conjugate \compound{cmpd:SHL4CipMe} showed better anti-biofilm activity than antibiotic, and it is possible that the other CipMe conjugates will do the same.

The most promising compounds from this set were the methoxybenzene-ciprofloxacin triazole conjugates \compound{cmpd:2MeOA4T4Cip} and \compound{cmpd:3MeOA4T4Cip} and the cleavable HCTL-Cip triazole conjugate \compound{cmpd:SHL4THCip}. 
The cleavable HCTL-Cip triazole conjugate \compound{cmpd:SHL4THCip} was also interesting in that it appeared to entirely inhibit the growth of \textit{P. aeruginosa}, whereas ciprofloxacin \compound{cmpd:Cip} either allowed slow growth, or resistant mutants emerged and started to replicate\cite{Su2010}.
However, the errors in these data were large and so the assays need to be repeated.
Given the previous results for the cleavable HSL-Cip conjugates (see \ref{sec:bioC}) is not very likely that the quorum sensing modulation properties of the head group contributed to its antibiotic effect, but the effect of different cleavable tails is certainly worth investigating.

Initial biofilm inhibition and dispersal assays were carried out (see \ref{sec:exp_bio}), but unfortunately the results were unreliable as the biofilm was growing more in the wells at the edges of the plates.
As biofilm formation is induced by hypoxia\cite{Ghotaslou2013} (which might occur in the centre of the plate when a plastic lid was used in addition to the adhesive plate seal) this cannot account for the increased biofilm growth at the edges of the plate. Neither can the increased concentration of NaCl that would occur due to evaporation of water from the edges of the plate, as this too decreases biofilm formation\cite{Bazire2007}. %\todo{culture conc?}
A reasonable explanation is that evaporation would leave a residue of dried planktonic cells on the edges of the wells which would be stained by the crystal violet.