Clicks

clicks were all crap because dilute Hong2009

SHL

Future optimisation of the synthesis could focus on different routes to the product, e.g. the peptide coupling described in \ref{sec:CipMe_linker}, or different purification methods, e.g. using just preparatory HPLC, or reverse phase flash column chromatography.



HOcy5

Direct comparisons of routes are not possible without repeating syntheses using this new method, but if it is assumed that peptide coupling of homocysteine thiolactone hydrochloride \compound{cmpd:SHLHCl} to carboxylic acid \compound{cmpd:HOO4CipMeTFA} would have a similar yield to the coupling with (1\textit{R},2\textit{R})-2-aminocyclopentan-1-ol \compound{cmpd:HOcy5NH2_RR}, approximate comparisons can be made.
The synthesis described in \ref{sec:HCTL} has an overall yield of 10.7 \%, whereas the route shown in \ref{sch:HOcy5NH4CipMe_RR_synth} for \compound{cmpd:HOcy5NH4CipMe_SS} has an overall yield of 26.1 \%. Moreover, if the yield starting from the head group (which may be expensive, difficult to synthesise and/or unstable) is considered, the yield is 54.7 \% vs. 10.7 \%.
Therefore, this route is recommended for further investigation if the library is to be expanded.

A downside to this route is that it cannot branch towards the triazole-coupled library in the same way that the route in \ref{sec:HCTL}. A carboxylic acid intermediate with a triazole in the chain could presumably be synthesised, but this would be rather pointless given that the triazole library was initially proposed so that the two sides could be joined by the click reaction.



No, I didn't try the one-pot synthesis without TBS.
No worries, I wonder if it would have worked. Could be one for the conclusions?


Not C4 chain - massive pain due to internal ring formation.

%From head
%87.9*12.2=10.7
%38.7
%longest route
%87.9*12.2=10.7
%49.9*95.6*54.7=26.1