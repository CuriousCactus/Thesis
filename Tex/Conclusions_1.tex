\subsection{Conclusions}

\subsubsection{Library synthesis}

In this section, a range of 1,2,3-triazole-linked autoinducer-antibiotic conjugates was successfully synthesised and tested for antibiotic and anti-biofilm activity.
Reliable routes to the azido autoinducers and alkynyl antibiotics were found, but the copper(I)-catalyzed alkyne-azide cycloaddition reactions used to link them proved rather capricious.
The main reasons for this were insolubility of the starting materials and air-sensitivity. 
Air-sensitivity is not expected in a click reaction, but can be explained by many of the reactions being too dilute\cite{Hong2009}. 
This led to ascorbate being used up by the oxygen dissolved in the reaction solvent and present in the air above the reaction mixture. 
Even when the solvent was degassed and the reaction performed under argon, a small amount of air leaking in through a perished septum was enough to cause the reaction to stall.
Low concentrations were used because of the insolubility of the starting materials, but this would have been better addressed by more thorough screening of solvents.
In addition, it was later shown that THPTA may not be necessary for a sufficiently concentrated reaction to take place\cite{Stokes2017}, so this expensive reagent could be omitted.

Assuming the click reaction could be further optimised, this library could be easily expanded by the addition of more azido autoinducers and alkynyl antibiotics (see \ref{sec:Fut1}). In particular, autoinducers which are actively transported into cells, such as AI-2, are attractive targets.

\subsubsection{Biology}

Little biological activity was seen in the non-cleavable autoinducer-antibiotic conjugates. This could be due to a number of factors, including:

\begin{enumerate}
\item Restriction of the binding of ciprofloxacin to DNA gyrase and topoisomerase IV\cite{Drlica1997} or trimethoprim to dihydrofolate reductase\cite{Brogden1982}.

\item Failure of the autoinducers to mask the antibiotics from recognition by efflux pumps (PAO1 only).

\item Failure to penetrate the cell wall/biofilm.

\item Non-specific binding to the cell wall.
\end{enumerate}

If binding of the antibiotics to target proteins is indeed restricted by the attachment of the autoinducer, this could be affected by the size and polarity of the linker and autoinducer. With this in mind, the next set of compounds synthesised contain HSL analogues, which are smaller than HHQ \compound{cmpd:HHQ} and PQS \compound{cmpd:PQS}, and some omit the triazole in the linker, hence affecting polarity.

The cleavable HSL-ciprofloxacin conjugates showed a little more activity, but unfortunately this did not require the HSL, and probably was mostly affected by the polarity and size of the attached group and the ease of hydrolysis of the linker.