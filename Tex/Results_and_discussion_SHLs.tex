\subsection{Homocysteine thiolactone derivatives\label{sec:HCTL}}

\subsubsection{Synthesis of methyl ciprofloxacin \compound{cmpd:CipMe}}

The synthesis of the analogue conjugates began with the synthesis of methyl ciprofloxacin \compound{cmpd:CipMe} (CipMe), which would then be attached to the various head groups.
Methyl ciprofloxacin \compound{cmpd:CipMe} was synthesised from ciprofloxacin \compound{cmpd:Cip} and MeOH in very good yield using \textit{para}-toluenesulfonic acid as a catalyst \cite{Sachin2010}.

\begin{scheme}[H]
	\begin{center}
		\schemeref[Cip]{cmpd:Cip}	
		\schemeref[CipMe]{cmpd:CipMe}
		\includegraphics[scale=1]{CipMe_synth}
		\caption{Synthesis of methyl ciprofloxacin \compound{cmpd:CipMe}. a) TsOH, MeOH, 72 h, reflux, 83.3 \%. \label{sch:CipMe_synth}}
	\end{center}
\end{scheme}

\subsubsection{Synthesis of Br-C$_4$-HCTL \compound{cmpd:SHL4Br}}

The HCTL head group was then attached to the linker to form Br-C$_4$-HCTL \compound{cmpd:SHL4Br}, in preparation for coupling to methyl ciprofloxacin \compound{cmpd:CipMe}.
Br-C$_4$-HCTL \compound{cmpd:SHL4Br} was synthesised using the Schotten-Baumann conditions employed previously for the HSL derivatives \compound{cmpd:HL4Br} and \compound{cmpd:HL6Br}. Br-C$_4$-HCTL \compound{cmpd:SHL4Br} was isolated in markedly higher yield than that achieved by Ganguly \textit{et al.}\cite{Ganguly2011} (87.9 \% vs. 25.0 \%). It is possible that this was due to \ce{CH2Cl2} being used for the extraction, whereas Ganguly \textit{et al.} used EtOAc.

\begin{scheme}[H]
	\begin{center}
		\schemeref[Cl4Br]{cmpd:Cl4Br}
		\schemeref[SHLHCl]{cmpd:SHLHCl}	
		\schemeref[SHL4Br]{cmpd:SHL4Br}
		\includegraphics[scale=1]{SHL4Br_synth}
		\caption{Synthesis of Br-C$_4$-HCTL \compound{cmpd:SHL4Br}. a) \ce{NaHCO3}, \ce{CH2Cl2}, \ce{H2O}, 0 $^{\circ}$C, 1 h, 87.9 \%.\label{sch:SHL4Br_synth}}
	\end{center}
\end{scheme}

\subsubsection{Synthesis of the HCTL-CipMe conjugate \compound{cmpd:SHL4CipMe}}

The HCTL-CipMe conjugate \compound{cmpd:SHL4CipMe} was synthesised using the procedure outlined by Ganguly \textit{et al.}\cite{Ganguly2011}. Monitoring by LCMS showed slow conversion to the product. Br-C$_4$-HCTL \compound{cmpd:SHL4Br} was presumably consumed by side reactions as 4 eq. were required to reach full conversion. Ganguly \textit{et al.} do not quote a yield for comparison\cite{Ganguly2011,Iyer2012}, but it is hoped that the 12.2 \% achieved here could be improved upon. The side reactions led to the production of an unidentified brown, viscous contaminant which made purification by flash column chromatography (as was used by Ganguly \textit{et al.}) challenging. Preparatory HPLC on a partially purified sample gave enough pure HCTL-CipMe conjugate \compound{cmpd:SHL4CipMe} for biological testing. 

Future optimisation of the synthesis could focus on different routes to the product, e.g. the peptide coupling described in \ref{sec:CipMe_linker}, or different purification methods, e.g. using just preparatory HPLC, or reverse phase flash column chromatography.

\begin{scheme}[H]
	\begin{center}
		\schemeref[CipMe]{cmpd:CipMe}
		\schemeref[SHL4CipMe]{cmpd:SHL4CipMe}
		\schemeref[SHL4Br]{cmpd:SHL4Br}
		\includegraphics[scale=1]{SHL4CipMe_synth}
		\caption{
			Synthesis of the HCTL-CipMe conjugate \compound{cmpd:SHL4CipMe}, 
			\ce{N3}-C$_4$-HCTL \compound{cmpd:SHL4N3}, and
			the HCTL-Cip triazole conjugate \compound{cmpd:SHL4T4Cip}.
			a) \ce{K2CO3}, acetonitrile, reflux, 24 h, 12.2 \%.
			\label{sch:SHL4CipMe_synth}}
	\end{center}
\end{scheme}

\subsubsection{Synthesis of the HCTL-Cip triazole conjugate \compound{cmpd:SHL4T4Cip}}

Br-C$_4$-HCTL \compound{cmpd:SHL4Br} was converted into \ce{N3}-C$_4$-HCTL \compound{cmpd:SHL4N3} (see \ref{sch:SHL4CipMe_synth}), by an S$_N$2 reaction with sodium azide which proceeded in excellent yield. 

\ce{N3}-C$_4$-HCTL \compound{cmpd:SHL4N3} was then subjected to the click reaction conditions optimised previously (see \ref{sec:click_general}). The reaction proceeded very slowly at first, until it was realised that the azide did not dissolve in the reaction solvent and formed a single solid clump. DMSO was added as a co-solvent, and the reaction began to proceed, albeit still slowly. It is possible that the sulfur atom coordinates to the copper, thus inhibiting its catalytic ability. Nonetheless the HCTL-Cip triazole conjugate \compound{cmpd:SHL4T4Cip} was eventually isolated in good yield (see \ref{sch:SHL4T4Cip_synth}).

\begin{scheme}[H]
	\begin{center}
		\schemeref[SHL4Br]{cmpd:SHL4Br}
		\schemeref[SHL4N3]{cmpd:SHL4N3}
		\schemeref[Y4Cip]{cmpd:Y4Cip}
		\schemeref[SHL4T4Cip]{cmpd:SHL4T4Cip}
		\includegraphics[scale=1]{SHL4T4Cip_synth}
		\caption{
			Synthesis of the HCTL-Cip triazole conjugate \compound{cmpd:SHL4T4Cip}.
			a) \ce{NaN3}, acetonitrile, reflux, 1.5 h, 89.3 \%.
			b) \ce{CuSO4}, THPTA, sodium ascorbate, \ce{H2O}, \textit{t}-BuOH, DMSO, r.t., 7 d, 70.6 \%.
			\label{sch:SHL4T4Cip_synth}}
	\end{center}
\end{scheme}

\subsubsection{Synthesis of the cleavable HCTL-Cip triazole conjugate \compound{cmpd:SHL4THCip}}

A cleavable conjugate \compound{cmpd:SHL4THCip} was also synthesised from \ce{N3}-C$_4$-HCTL \compound{cmpd:SHL4N3} by reaction with a cleavable alkyne-Cip derivative \compound{cmpd:Y4HCip} synthesised previously by Prof. Eddy Sotelo-Perez (see \ref{sec:cleavable}). Conditions developed by Prof. Sotelo-Perez were used, but again the reaction proceeded very slowly. The disappointing yield is, however, most likely due to losses during purification.

\begin{scheme}[H]
	\begin{center}
		\schemeref[SHL4N3]{cmpd:SHL4N3}
		\schemeref[Y4HCip]{cmpd:Y4HCip}
		\schemeref[SHL4THCip]{cmpd:SHL4THCip}
		\includegraphics[scale=1]{SHL4THCip_synth}
		\caption{
			Synthesis of the cleavable HCTL-Cip triazole conjugate \compound{cmpd:SHL4THCip}.
			a) CuI, DIPEA, \ce{CH2Cl2}, r.t., 3 h, 5.0 \%.
			\label{sch:SHL4THCip_synth}}
	\end{center}
\end{scheme}