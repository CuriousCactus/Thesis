\section{Declaration}

This dissertation describes work carried out in the Department of Chemistry, University of Cambridge under the supervision of Prof. David Spring, and in the Department of Biochemistry, University of Cambridge under the supervision of Dr Martin Welch. 
This dissertation is the result of my own work and includes nothing that is the outcome of work done in collaboration except as specified in the text. The dissertation does not exceed the word limit specified by the Physics and Chemistry Degree Committee.

\vspace{5cm}



Lois Overvoorde
September 2018
\newpage

\section{Abstract}

Bacterial resistance to antibiotics is becoming a serious global health threat, and the discovery of new, safe and effective antibiotics is required urgently\cite{ResistanceUS,davies2013drugs,ANIE:ANIE201209979}. A new class of antibiotic, namely sideophore-antibiotic conjugates, has shown promise in initial studies\cite{Page2013,Schalk2017}. Siderophores are used by bacteria for iron uptake, and so attaching antibiotics to them allows the antibiotic to be carried across cell membranes. This study investigates conjugates designed using a similar approach, but using bacterial autoinducers\cite{Waters2005} instead of siderophores. Autoinducers are required for coordination of bacterial behaviours and are involved in the control of swarming, virulence factor production and biofilm formation\cite{Miller2001}. 


The library was synthesised in two halves which were then coupled together using a copper(I)-catalysed azide-alkyne cycloaddition\cite{Tornoe2002,ANIE:ANIE2596}. The autoinducers were functionalised with azide groups and the antibiotics were functionalised with alkynes. The quorum sensing molecules produced by \textit{Pseudomonas aeruginosa} were investigated as it is a significant human pathogen\cite{Bodey1983} which displays high resistance to many antibiotics\cite{Poole2004} and uses quorum sensing to coordinate its group behaviours\cite{Dubern2008}. Azido analogues of these autoinducers were coupled with alkyne analogues of ciprofloxacin, which was chosen as it is commonly used against \textit{P. aeruginosa}\cite{Macgowan1999} but resistance to it is developing\cite{Su2010}, and trimethoprim. It was hoped that the autoinducers would aid retention of the antibiotic in the cell, thus potentially increasing its potency or even restoring its efficacy against resistant strains.
\todo{analogues,cleavables,bio}

\newpage