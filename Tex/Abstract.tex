\section{Abstract}

Bacterial resistance to antibiotics is becoming a serious global health threat, and the discovery of new, safe and effective antibiotics is required urgently\cite{ResistanceUS,davies2013drugs,ANIE:ANIE201209979}. A new class of antibiotic, namely sideophore-antibiotic conjugates, has shown promise in initial studies\cite{Page2013}. Siderophores are used by bacteria for iron uptake, and so attaching antibiotics to them allows the antibiotic to be carried across cell membranes. We have designed conjugates using a similar approach, but using bacterial quorum sensing molecules\cite{Waters2005} instead of siderophores. Quorum sensing molecules are required for coordination of bacterial behaviours and are involved in the control of swarming, virulence factor production and biofilm formation. 

The library is synthesised in two halves which are then coupled together using a copper(I)-catalysed azide-alkyne cycloaddition\cite{Tornoe2002,ANIE:ANIE2596}. The quorum sensing molecule analogues have azide groups attached and the antibiotic analogues have alkynes attached. We decided to focus on the quorum sensing molecules produced by \textit{Pseudomonas aeruginosa} as it is a significant human pathogen\cite{Bodey1983} which displays high resistance to many antibiotics\cite{Poole2004} and uses quorum sensing to coordinate its group behaviours\cite{Dubern2008}. Azido analogues of these quorum sensing molecules will be coupled with alkyne analogues of ciprofloxacin, which was chosen as it is commonly used against \textit{P. aeruginosa}\cite{Macgowan1999} but resistance to it is developing\cite{Su2010}. It is hoped that the quorum sensing molecules will deliver the attached ciprofloxacin into the cell, thus potentially increasing its potency or even restoring its efficacy against resistant strains.

\newpage