\newpage

\section{Declaration}

This dissertation describes work carried out in the Department of Chemistry, University of Cambridge under the supervision of Professor David Spring, and in the Department of Biochemistry, University of Cambridge under the supervision of Dr Martin Welch. 
This dissertation is the result of my own work and includes nothing that is the outcome of work done in collaboration except as specified in the text. 
It is not substantially the same as any that I have submitted, or, is being concurrently submitted for a degree or diploma or other qualification at the University of Cambridge or any other University or similar institution. 
I further state that no substantial part of my dissertation has already been submitted, or, is being concurrently submitted for any such degree, diploma or other qualification at the University of Cambridge or any other University or similar institution, except those parts which were included in my CPGS dissertation.
The dissertation does not exceed the word limit specified by the Physics and Chemistry Degree Committee.








\vspace{5cm}



Lois Overvoorde

7th of September 2018

\newpage

\section{Abstract}

Microbial resistance to antibiotics is a serious global health threat, and the discovery of new, safe and effective antibiotics is required urgently\cite{ResistanceUS,davies2013drugs,ANIE:ANIE201209979}. A new class of antibiotics, namely sideophore-antibiotic conjugates, has shown promise in initial studies\cite{Page2013,Schalk2017}. Siderophores are used by bacteria for iron uptake, and so attaching antibiotics to them allows the antibiotic to be carried across cell membranes. This study investigated conjugates designed using a similar approach, but using bacterial autoinducers\cite{Waters2005} instead of siderophores. Autoinducers are required for coordination of bacterial behaviours and are involved in the control of swarming, virulence factor production and biofilm formation\cite{Miller2001}. 

The quorum sensing molecules produced by \textit{Pseudomonas aeruginosa} were chosen for investigation as \textit{P. aeruginosa} is a significant human pathogen\cite{Bodey1983} which displays high resistance to many antibiotics\cite{Poole2004} and uses quorum sensing to coordinate its group behaviours\cite{Dubern2008}. 
Ciprofloxacin and trimethoprim were chosen as the antibiotic partners.
Ciprofloxacin is commonly used against \textit{P. aeruginosa}\cite{Macgowan1999} but resistance to it is developing\cite{Su2010}, whereas \textit{P. aeruginosa} is inherently resistant to trimethoprim.
It was hypothesised that the autoinducers would aid retention of the antibiotics in cells, hence increasing or restoring activity.

An initial library was synthesised in two halves which were coupled together using a copper(I)-catalysed azide-alkyne cycloaddition\cite{Tornoe2002,Rostovtsev2002}. 
The autoinducers were functionalised with azide groups and the antibiotics (specifically ciprofloxacin and trimethoprim) were functionalised with alkynes. 
Two cleavable alkynyl ciprofloxacin derivatives were also included.

A second set of compounds, namely homoserine lactone analogue-ciprofloxacin conjugates were then synthesised, building on the one known report of a conjugate of a quorum sensing modulator and an antibiotic\cite{Ganguly2011}.

The most active conjugate found was a cleavable conjugate of homocysteine thiolactone (a homoserine lactone analogue) and ciprofloxacin. This compound showed enhanced antibacterial activity against \textit{P. aeruginosa} compared to ciprofloxacin, and \textit{P. aeruginosa} may develop less resistance towards it.


\newpage