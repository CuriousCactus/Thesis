\newpage

\section{Experimental}

\subsection{General}

Unless otherwise stated, reactions were performed in air-dried glassware under argon with dry, freshly distilled solvents. THF was distilled from \ce{LiAlH4} in the presence of triphenyl methane indicator. \ce{CH2Cl2}, hexane, MeOH and MeCN were distilled from calcium hydride. All other chemicals were used as obtained from commercial sources.

Reactions using microwave heating were performed in sealed vials using a CEM Discover SP microwave reactor. 

Thin Layer Chromatography (TLC) was performed using Merck pre-coated 0.23 mm thick plates of Keiselgel 60 F254 and visualised using UV ($\lambda$ = 254 or 366 nm) or by staining with \ce{KMnO4} or ninhydrin. 
All retention factors (\textit{R}$_\textit{f}$) are given to 0.01. 
All column chromatography was carried out using Merck 9385 Keiselgel 60 silica gel (230-400 mesh) or using a CombiFlash$^{\tiny{\textregistered}}$ EZ Prep with RediSep$^{\tiny{\textregistered}}$ normal-phase silica flash columns. 
Preparative High Pressure Liquid Chromatography (prep. HPLC) was run on an Agilent 1260 Infinity machine, using a Supelcosil\textsuperscript{TM} ABZ+PLUS column (250 mm $\times$ 21.2 mm, 5 $\mu$m) with a linear gradient system (solvent A: 0.1 \% ($v$/$v$) TFA/water, solvent B: 0.05 \% ($v$/$v$) TFA/acetonitrile) at a flow rate of 20 mL min$^{-1}$, visualised by UV absorbance ($\lambda_{max}$ = 254 nm)

Nuclear Magnetic Resonance (NMR) spectra were recorded using an internal deuterium lock at ambient probe temperatures on Bruker DPX-400, Bruker Avance DRX-400, Bruker Avance 500 BB-ATM or Bruker Avance 500 Cryo Ultrashield spectrometers. Data were processed using NMR Processor Academic Edition version 12 (ADC Labs) or TopSpin version 3.5 (Bruker). $^1$H and $^13$C spectra were assigned using DEPT, COSY, HMQC and HMQC spectra where necessary, or by analogy to fully interpreted spectra of related compounds. The following abbreviations are used to indicate the multiplicity of signals: s singlet, d doublet, t triplet, q quartet, quin quintet, m multiplet and br broad.

$^1$H chemical shifts ($\delta$) are quoted to the nearest 0.01 ppm and are referenced relative to the residual solvent peak\cite{Gottlieb1997}. Coupling constants (\textit{J}) are given to the nearest 0.1 Hz. Diastereotopic protons are assigned as C\underline{H}H and CH\underline{H},
where the latter designates the lower‑field proton. Data are reported as follows: <chemical shift> (<multiplicity>, <coupling constant(s) (if any)>, <integration>, <assignment>).

$^{13}$C chemical shifts ($\delta$) are quoted to the nearest 0.1 ppm and are referenced relative to the deuterated solvent peak\cite{Gottlieb1997}. Data are reported as follows: <chemical shift> (<multiplicity (if not s)>, <coupling constant(s) (if any)>, <assignment>).

$^{19}$F chemical shifts ($\delta$) are quoted to the nearest 0.1 ppm. Data are reported as follows: <chemical shift> (<assignment>).

High Resolution Mass Spectra (HRMS) were recorded using either a Micromass Q-TOF or a Micromass LCT Premier spectrometer and reported mass values are within $\pm$ 5 ppm mass units. Low Resolution Mass Spectra (LRMS) were recorded on an Agilent 1200 series LC with an ESCi Multi-Mode Ionisation Waters ZQ spectrometer or a Waters ACQUITY H-Class UPLC with an ESCi Multi-Mode Ionisation Waters SQ Detector 2 spectrometer.

Infra Red (IR) spectra were recorded using neat sample on a PerkinElmer 1600 FT IR spectrometer. Selected absorption maxima ($\nu_{max}$) are reported in wavenumbers (cm$^{-1}$). Broad peaks are marked br.

Melting points (m.p.) were measured using a Buchi B-545 melting point apparatus and are uncorrected. 

Optical rotations ([$\alpha$]$_D^T$) were recorded on a PerkinElmer 343 polarimeter or an Anton-Paar MCP 100 polarimeter. [$\alpha$]$_D^T$ values are reported in $^{\circ}$10$^{-1}$cm$^2$g$^{-1}$ at 589 nm and concentration (\textit{c}) is given in g (100 mL)$^{-1}$. 

%\subsection{NMR (from Sean)}
%
%Magnetic resonance spectra were processed using iNMR v. 5.5.7 (Mestrelab Research) or
%TopSpin v. 3.5 (Bruker). An aryl, quaternary, or two or more possible assignments were
%given when signals could not be distinguished by any means. Measured coupling
%constants are reported for mutually coupled signals; coupling constants are labelled
%apparent in the absence of an observed mutual coupling, or multiplet when none can
%be determined.
%Proton magnetic resonance spectra were recorded using an internal deuterium lock (at
%298 K unless stated otherwise) on Bruker DPX (400 MHz; 1H-13C DUL probe), Bruker
%Avance III HD (400 MHz; Smart probe), Bruker Avance III HD (500 MHz; Smart probe)
%and Bruker Avance III HD (500 MHz; DCH Cryoprobe) spectrometers. Proton assignments
%are supported by 1 H‑ 1 H COSY, 1 H‑ 13 C HSQC or 1 H‑ 13 C HMBC spectra, or by analogy.
%Chemical shifts ($\delta$ H ) are quoted in ppm to the nearest 0.01 ppm and are referenced to
%the residual non‑deuterated solvent peak. Discernable coupling constants for mutually
%coupled protons are reported as measured values in Hertz, rounded to the nearest
%0.1 Hz. Data are reported as: chemical shift, number of nuclei, multiplicity (br, broad; s,
%singlet; d, doublet; t, triplet; q, quartet; m, multiplet; or a combination thereof),
%coupling constants and assignment. Diastereotopic protons are assigned as X and X’,
%where X’ designates the lower‑field proton.
%Carbon magnetic resonance spectra were recorded using an internal deuterium lock (at
%298 K unless stated otherwise) on Bruker DPX (101 MHz), Bruker Avance III HD
%(101 MHz) and Bruker Avance III HD (126 MHz) spectrometers with broadband proton
%decoupling. Carbon spectra assignments are supported by DEPT editing, 1 H‑ 13 C HSQC or
%1 H‑ 13 C HMBC spectra, or by analogy. Chemical shifts ($\delta$ C ) are quoted in ppm to the
%nearest 0.1 ppm and are referenced to the deuterated solvent peak. Data are reported
%as: chemical shift, number of nuclei (if not one), multiplicity (if not a singlet), coupling
%constants and assignment.
%Fluorine magnetic resonance spectra were recorded on Bruker Avance III (376 MHz;
%QNP Cryoprobe) or Bruker Avance III HD (376 MHz; Smart probe) spectrometers.
%Chemical shifts ($\delta$ F ) are quoted in ppm to the nearest 0.1 ppm. Data are reported as:
%chemical shift, number of nuclei (if not one), multiplicity (if not a singlet), coupling
%constants and assignment.

%Every new compound should be
%characterised fully for publication/thesis, and this involves: 1

%%H, 13C NMR, IR, MS and either microanalysis (within 0.3%) or HRMS (within 5 ppm). Also don’t forget a melting point (if a solid)
%and optical rotation (if chiral). If the compound has been reported in the literature before, then
%microanalysis and HRMS are not necessary, but do make sure the other data is consistent with
%the reported data (especially melting point and optical rotation). The NMR spectra for every
%significant compound reported in your thesis (CPGS/PhD) should be included in an appendix (A4)
%labelled with your code number and the compound structure.

