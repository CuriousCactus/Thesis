\newpage

\section{Experimental}

\subsection{General}

Unless otherwise stated, reactions were performed in oven-dried glassware under argon with dry, freshly distilled solvents. THF was distilled from \ce{LiAlH4} in the presence of triphenyl methane indicator. \ce{CH2Cl2}, hexane, MeOH, pyridine and MeCN were distilled from calcium hydride. Anhydrous DMF (Acros) and 1,2-dichloroethane (Aldrich) were used without further purification. All other chemicals were used as obtained from commercial sources.

Unless otherwise stated, yields refer to analytically pure compounds. TLC was performed using Merck pre-coated 0.23 mm thick plates of Keiselgel 60 F254 and visualised using UV ($\lambda$ = 254 nm) or by staining with \ce{KMnO4}. 
All retention factors (\textit{R}$_\textit{f}$) are given to 0.05. 
All column chromatography was carried out using Merck 9385 Keiselgel 60 silica gel (230-400 mesh).

$^1$H NMR spectra were recorded on a Bruker DPX 400 spectrometer operating at 400 MHz using an internal deuterium lock at ambient probe temperatures. Chemical shifts ($\delta$) are quoted to the nearest 0.01 ppm and are referenced relative to residual solvent peak\cite{Gottlieb1997}. Coupling constants (\textit{J}) are given to the nearest 0.1 Hz. The following abbreviations are used to indicate indicate the multiplicity of signals: s singlet, d doublet, t triplet, q quartet, m multiplet and br broad. Data is reported as follows: chemical shift (multiplicity, coupling constant(s), integration, assignment).

$^{13}$C NMR spectra were recorded on a Bruker 400 spectrometer operating at 101 MHz using an internal deuterium lock at ambient probe temperatures. Chemical shifts ($\delta$) are quoted to the nearest 0.1 ppm and are referenced relative to the deuterated solvent peak\cite{Gottlieb1997}. NMR assignments are supported by DEPT editing and where necessary COSY, HMQC, and HMQC. 

High resolution mass spectra (HRMS) were recorded using either a Micromass Q-TOF or a Micromass LCT Premier spectrometer and reported mass values are within $\pm$ 5 ppm mass units unless otherwise stated. Low resolution mass spectra (LRMS) spectra were recorded on an ACQUITY UPLC with an ESCi Multi-Mode Ionisation Waters Zspray spectrometer using MassLynx 4.1 software.

IR spectra were recorded using neat sample on a PerkinElmer 1600 FT IR spectrometer. Selected absorption maxima ($\nu_{max}$) are reported in wavenumbers (cm$^{-1}$). Broad peaks are marked br.
%The following abbreviations are used: s strong, m medium, w weak and br broad.
Melting points (m.p.) were measured using a Buchi B-545 melting point apparatus and are uncorrected. 
Optical rotations ([$\alpha$]$_D$) were recorded on a PerkinElmer 343 polarimeter. [$\alpha$]$_D$ values are reported in $^{\circ}$10$^{-1}$cm$^2$g$^{-1}$ at 589 nm and concentration (\textit{c}) is given in g (100 mL)$^{-1}$. 

\subsection{NMR (from Sean)}

Magnetic resonance spectra were processed using iNMR v. 5.5.7 (Mestrelab Research) or
TopSpin v. 3.5 (Bruker). An aryl, quaternary, or two or more possible assignments were
given when signals could not be distinguished by any means. Measured coupling
constants are reported for mutually coupled signals; coupling constants are labelled
apparent in the absence of an observed mutual coupling, or multiplet when none can
be determined.
Proton magnetic resonance spectra were recorded using an internal deuterium lock (at
298 K unless stated otherwise) on Bruker DPX (400 MHz; 1H-13C DUL probe), Bruker
Avance III HD (400 MHz; Smart probe), Bruker Avance III HD (500 MHz; Smart probe)
and Bruker Avance III HD (500 MHz; DCH Cryoprobe) spectrometers. Proton assignments
are supported by 1 H‑ 1 H COSY, 1 H‑ 13 C HSQC or 1 H‑ 13 C HMBC spectra, or by analogy.
Chemical shifts ($\delta$ H ) are quoted in ppm to the nearest 0.01 ppm and are referenced to
the residual non‑deuterated solvent peak. Discernable coupling constants for mutually
coupled protons are reported as measured values in Hertz, rounded to the nearest
0.1 Hz. Data are reported as: chemical shift, number of nuclei, multiplicity (br, broad; s,
singlet; d, doublet; t, triplet; q, quartet; m, multiplet; or a combination thereof),
coupling constants and assignment. Diastereotopic protons are assigned as X and X’,
where X’ designates the lower‑field proton.
Carbon magnetic resonance spectra were recorded using an internal deuterium lock (at
298 K unless stated otherwise) on Bruker DPX (101 MHz), Bruker Avance III HD
(101 MHz) and Bruker Avance III HD (126 MHz) spectrometers with broadband proton
decoupling. Carbon spectra assignments are supported by DEPT editing, 1 H‑ 13 C HSQC or
1 H‑ 13 C HMBC spectra, or by analogy. Chemical shifts ($\delta$ C ) are quoted in ppm to the
nearest 0.1 ppm and are referenced to the deuterated solvent peak. Data are reported
as: chemical shift, number of nuclei (if not one), multiplicity (if not a singlet), coupling
constants and assignment.
Fluorine magnetic resonance spectra were recorded on Bruker Avance III (376 MHz;
QNP Cryoprobe) or Bruker Avance III HD (376 MHz; Smart probe) spectrometers.
Chemical shifts ($\delta$ F ) are quoted in ppm to the nearest 0.1 ppm. Data are reported as:
chemical shift, number of nuclei (if not one), multiplicity (if not a singlet), coupling
constants and assignment.

Every new compound should be
characterised fully for publication/thesis, and this involves: 1

H, 13C NMR, IR, MS and either
microanalysis (within 0.3%) or HRMS (within 5 ppm). Also don’t forget a melting point (if a solid)
and optical rotation (if chiral). If the compound has been reported in the literature before, then
microanalysis and HRMS are not necessary, but do make sure the other data is consistent with
the reported data (especially melting point and optical rotation). The NMR spectra for every
significant compound reported in your thesis (CPGS/PhD) should be included in an appendix (A4)
labelled with your code number and the compound structure.

